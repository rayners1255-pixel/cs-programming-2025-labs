\documentclass[]{vvsu}

\vvsuyear{2026}

\usepackage{graphicx}
\usepackage{tabularray}
\usepackage{listings}
\usepackage{xcolor}
\usepackage{caption}
\usepackage{adjustbox}
\usepackage{array}
\usepackage{longtable}
\usepackage{enumitem}

% Настройка для таблиц
\NewTblrTheme{vvsu}{
  \SetTblrStyle{head}{font=\bfseries}
  \SetTblrStyle{caption-tag}{font=\bfseries}
}

\SetTblrInner[vvsu]{
  hlines = {0.8pt, white},
  vlines = {0.8pt, white},
  hline{1,Z} = {1pt},
  vline{1,Z} = {1pt},
  row{1} = {font=\bfseries},
  width = \linewidth,
  colsep = 6pt,
  rows = {rowsep = 0pt}
}

\graphicspath{{images/}}

% Файл со списком источников
\addbibresource{references.bib}

\author{Р.А. Сакс}

\begin{document}

\vvsuhead{\linespread{1}\selectfont{}МИНОБРНАУКИ РОССИИ\\
\vspace{10pt}Федеральное государственное бюджетное образовательное учреждение\\
высшего образования\\
\fontsize{13}{13}\selectfont{}<<ВЛАДИВОСТОКСКИЙ ГОСУДАРСТВЕННЫЙ УНИВЕРСИТЕТ>>\\
(ФГБОУ ВО <<ВВГУ>>)\\
\vspace{10pt}\fontsize{12}{12}\selectfont{}ИНСТИТУТ ИНФОРМАЦИОННЫХ ТЕХНОЛОГИЙ И АНАЛИЗА ДАННЫХ\\
КАФЕДРА ИНФОРМАЦИОННЫХ ТЕХНОЛОГИЙ И СИСТЕМ}

\title{ОТЧЕТ ПО ПРОЕКТУ}
\subtitle{Система управления автозаправочной станцией}

\vvsurecommendation{Руководитель проекта\\
ст. преподаватель}{М.В. Водяницкий}



\member{Студент\\ гр. БИН-25-3}{Р.А. Сакс}
\member{Руководитель проекта\\ ст. преподаватель}{М.В. Водяницкий}

\maketitle

\begin{addition}{Аннотация}
Данный отчет представляет результаты разработки консольной системы управления автозаправочной станцией (АЗС). Система реализована на языке Python и предназначена для автоматизации основных процессов работы АЗС, включая обслуживание клиентов, контроль запасов топлива, управление оборудованием и ведение статистики.

Система обеспечивает работу с различными типами топлива, управление цистернами и заправочными колонками, обработку транзакций, а также включает механизмы безопасности и аварийного отключения. Все данные сохраняются в файловой системе в формате JSON, что обеспечивает сохранение состояния между сеансами работы.

Разработанная система демонстрирует практическое применение принципов объектно-ориентированного программирования, работы с файлами и построения консольных интерфейсов.
\end{addition}

\begin{addition}{Abstract}
This report presents the results of developing a console-based automated fuel station management system. The system is implemented in Python and is designed to automate the main processes of a fuel station, including customer service, fuel inventory control, equipment management, and statistics keeping.

The system supports various fuel types, manages storage tanks and fuel dispensers, processes transactions, and includes safety mechanisms and emergency shutdown procedures. All data is stored in the file system in JSON format, ensuring state persistence between sessions.

The developed system demonstrates the practical application of object-oriented programming principles, file handling, and console interface design.
\end{addition}

\begin{addition}{Задание на проект}
Разработать консольную систему управления автозаправочной станцией, которая должна обеспечивать следующие функции:

\begin{enumerate}[label=\arabic*), leftmargin=*, itemsep=0pt]
\item Обслуживание клиентов (продажа топлива):
\begin{itemize}[label=--, leftmargin=*]
\item Выбор заправочной колонки
\item Выбор типа топлива
\item Расчет стоимости
\item Проверка доступности топлива
\item Фиксация транзакции
\end{itemize}

\item Управление цистернами:
\begin{itemize}[label=--, leftmargin=*]
\item Мониторинг уровня топлива
\item Автоматическое отключение при низком уровне
\item Ручное включение/отключение
\item Пополнение запасов
\end{itemize}

\item Управление оборудованием:
\begin{itemize}[label=--, leftmargin=*]
\item Контроль состояния заправочных колонок
\item Управление схемой подключения цистерн к колонкам
\end{itemize}

\item Статистика и отчетность:
\begin{itemize}[label=--, leftmargin=*]
\item Учет проданного топлива
\item Финансовая статистика
\item История операций
\end{itemize}

\item Безопасность:
\begin{itemize}[label=--, leftmargin=*]
\item Аварийный режим
\item Резервное копирование данных
\item Восстановление состояния
\end{itemize}

\item Технические требования:
\begin{itemize}[label=--, leftmargin=*]
\item Консольный интерфейс
\item Сохранение данных в файлы
\item Модульная архитектура
\item Обработка ошибок
\end{itemize}
\end{enumerate}
\end{addition}

% Ручное оглавление
{\centering\fontsize{14}{14}\linespread{1}\setmainfont{Arial}\selectfont Содержание\vspace{8pt}\\}

\noindent\textbf{Введение} \dotfill 3\\
\textbf{1 Анализ предметной области и проектирование системы} \dotfill 4\\
\hspace{0.5cm}1.1 Основные сущности и их характеристики \dotfill 4\\
\hspace{0.5cm}1.2 Схема подключения оборудования \dotfill 4\\
\hspace{0.5cm}1.3 Требования к функциональности \dotfill 5\\
\textbf{2 Архитектура и реализация системы} \dotfill 6\\
\hspace{0.5cm}2.1 Структура классов \dotfill 6\\
\hspace{0.5cm}2.2 Ключевые алгоритмы \dotfill 7\\
\hspace{0.5cm}2.3 Хранение данных \dotfill 8\\
\textbf{3 Реализация основных функций} \dotfill 9\\
\hspace{0.5cm}3.1 Модуль обслуживания клиентов \dotfill 9\\
\hspace{0.5cm}3.2 Модуль управления цистернами \dotfill 10\\
\hspace{0.5cm}3.3 Модуль статистики и отчетности \dotfill 11\\
\textbf{4 Тестирование и результаты} \dotfill 12\\
\hspace{0.5cm}4.1 Методология тестирования \dotfill 12\\
\hspace{0.5cm}4.2 Результаты тестирования \dotfill 12\\
\hspace{0.5cm}4.3 Обработка исключительных ситуаций \dotfill 13\\
\textbf{Заключение} \dotfill 14\\
\textbf{Список использованных источников} \dotfill 15\\
\textbf{Приложение А Исходный код программы} \dotfill 16\\
\textbf{Приложение Б Примеры работы системы} \dotfill 17

\clearpage

\begin{introduction}
Автоматизация процессов управления автозаправочными станциями является важной задачей в современных условиях развития топливно-энергетического комплекса. Эффективное управление запасами топлива, контроль за оборудованием и автоматизация расчетов с клиентами позволяют повысить производительность, снизить операционные издержки и минимизировать человеческие ошибки.

Целью данного проекта является разработка программной системы для управления автозаправочной станцией, которая позволит автоматизировать ключевые бизнес-процессы, обеспечить надежный учет операций и предоставить инструменты для анализа эффективности работы станции.

Основные задачи проекта:
\begin{enumerate}[label=\arabic*), leftmargin=*, itemsep=0pt]
\item Анализ предметной области и требований к системе управления АЗС
\item Проектирование архитектуры системы и структуры данных
\item Разработка классов для представления основных сущностей (цистерны, колонки, транзакции)
\item Реализация функционала обслуживания клиентов и управления оборудованием
\item Разработка механизмов сохранения и восстановления состояния системы
\item Создание консольного пользовательского интерфейса
\item Тестирование и отладка системы
\end{enumerate}

Система разработана на языке программирования Python, что обеспечивает кроссплатформенность, простоту поддержки и расширения функциональности.
\end{introduction}

\section{Анализ предметной области и проектирование системы}

\subsection{Основные сущности и их характеристики}

Автозаправочная станция представляет собой комплекс оборудования и бизнес-процессов, которые могут быть формализованы в виде следующих основных сущностей:

\begin{enumerate}[label=\arabic*), leftmargin=*, itemsep=0pt]
\item \textbf{Топливо} - характеризуется типом (АИ-92, АИ-95, АИ-98, ДТ) и ценой за литр
\item \textbf{Цистерна} - емкость для хранения топлива определенного типа, имеет максимальный объем, текущий уровень и состояние (включена/отключена)
\item \textbf{Заправочная колонка} - оборудование для отпуска топлива клиентам, может быть подключена к нескольким цистернам
\item \textbf{Транзакция} - запись о совершенной операции (продажа, пополнение, перекачка и т.д.)
\item \textbf{Статистика} - накопленные данные о работе АЗС
\end{enumerate}

\subsection{Схема подключения оборудования}

На АЗС реализована сложная схема подключения цистерн к заправочным колонкам, что обеспечивает гибкость в управлении топливными запасами. Каждая колонка может быть подключена к нескольким цистернам, а каждая цистерна может обслуживать несколько колонок. Это позволяет:

\begin{enumerate}[label=\arabic*), leftmargin=*, itemsep=0pt]
\item Распределять нагрузку между цистернами одного типа топлива
\item Обеспечивать бесперебойную работу при отключении одной из цистерн
\item Оптимизировать использование оборудования
\end{enumerate}

\subsection{Требования к функциональности}

На основе анализа технического задания были выделены следующие функциональные требования к системе:

\begin{table}[H]
\centering
\caption{Функциональные требования к системе}
\label{table:functional_requirements}
\begin{tblr}[
  theme = vvsu,
  ]{
    width = 0.95\linewidth,
    colspec = {|X[1.5,l]|X[3,l]|},
  }
Требование & Описание \\
Обслуживание клиентов & Продажа топлива с выбором колонки, типа топлива и количества литров \\
Управление цистернами & Контроль уровня топлива, автоматическое отключение при низком уровне, ручное управление \\
Пополнение запасов & Оформление поступления топлива от бензовоза \\
Статистика и отчеты & Учет продаж, доходов, количества обслуженных клиентов \\
История операций & Ведение журнала всех транзакций \\
Перекачка топлива & Перемещение топлива между цистернами одного типа \\
Аварийный режим & Блокировка всех операций при возникновении аварийной ситуации \\
Сохранение данных & Сохранение состояния системы между сеансами работы \\
\end{tblr}
\end{table}

\section{Архитектура и реализация системы}

\subsection{Структура классов}

Система построена на основе объектно-ориентированного подхода. Основные классы программы:

\begin{enumerate}[label=\arabic*), leftmargin=*, itemsep=0pt]
\item \textbf{Cistern (Цистерна)} - отвечает за хранение и управление запасами топлива
\item \textbf{FuelColumn (Заправочная колонка)} - управляет доступными типами топлива на колонке
\item \textbf{Transaction (Транзакция)} - представляет запись о совершенной операции
\item \textbf{Statistics (Статистика)} - накапливает и хранит данные о работе АЗС
\item \textbf{GasStation (АЗС)} - основной класс, объединяющий все компоненты системы
\end{enumerate}

\subsection{Ключевые алгоритмы}

\subsubsection{Алгоритм обслуживания клиента}

\begin{enumerate}[label=\arabic*), leftmargin=*, itemsep=0pt]
\item Выбор заправочной колонки из доступных
\item Проверка доступных типов топлива на выбранной колонке
\item Выбор типа топлива и проверка доступности в подключенной цистерне
\item Ввод количества литров и проверка достаточности запаса
\item Расчет стоимости на основе установленных цен
\item Подтверждение операции и списание топлива
\item Фиксация транзакции и обновление статистики
\end{enumerate}

\subsubsection{Алгоритм управления цистернами}

\begin{enumerate}[label=\arabic*), leftmargin=*, itemsep=0pt]
\item Регулярная проверка уровня топлива во всех цистернах
\item Автоматическое отключение цистерны при достижении минимального уровня
\item Предупреждение оператора о низком уровне топлива
\item Возможность ручного включения/отключения цистерн
\item Контроль при пополнении (недопущение переполнения)
\end{enumerate}

\subsection{Хранение данных}

Для обеспечения сохранности данных между сеансами работы системы реализован механизм сериализации в формат JSON. Сохраняются следующие данные:

\begin{enumerate}[label=\arabic*), leftmargin=*, itemsep=0pt]
\item Состояние всех цистерн (текущий объем, состояние включения)
\item Статистика продаж и транзакций
\item Состояние аварийного режима
\end{enumerate}

Данные сохраняются в файл \texttt{gas\_station\_data.json} при каждом изменении состояния системы.

\section{Реализация основных функций}

\subsection{Модуль обслуживания клиентов}

Реализация функции обслуживания клиента включает следующие этапы:

\begin{enumerate}[label=\arabic*), leftmargin=*, itemsep=0pt]
\item Проверка активности аварийного режима
\item Отображение списка доступных заправочных колонок
\item Получение выбора пользователя и проверка корректности
\item Определение доступных типов топлива на выбранной колонке
\item Проверка состояния подключенных цистерн
\item Ввод количества литров и проверка достаточности запаса
\item Расчет стоимости на основе установленных цен
\item Запрос подтверждения операции у пользователя
\item Списание топлива и обновление статистики
\item Фиксация транзакции в истории операций
\end{enumerate}

\subsection{Модуль управления цистернами}

Класс \texttt{Cistern} реализует следующую логику управления:

\begin{enumerate}[label=\arabic*), leftmargin=*, itemsep=0pt]
\item Инициализация с параметрами: название, тип топлива, максимальный объем
\item Автоматическая установка начального уровня (50\% от максимального)
\item Контроль минимального уровня (10\% от максимального объема)
\item Проверка возможности отпуска топлива с учетом состояния и запасов
\item Автоматическое отключение при достижении минимального уровня
\item Контроль при пополнении (проверка на переполнение)
\item Сериализация и десериализация состояния для сохранения в файл
\end{enumerate}

\subsection{Модуль статистики и отчетности}

Система ведет детальную статистику по всем аспектам работы АЗС:

\begin{enumerate}[label=\arabic*), leftmargin=*, itemsep=0pt]
\item Общий доход от продаж топлива
\item Количество обслуженных автомобилей
\item Детальная статистика по каждому типу топлива:
\begin{itemize}[label=--, leftmargin=*]
\item Количество проданных литров
\item Полученный доход
\item Количество транзакций
\end{itemize}
\item История последних 100 операций с timestamp
\item Ограничение размера истории для предотвращения переполнения памяти
\end{enumerate}

\section{Тестирование и результаты}

\subsection{Методология тестирования}

Для проверки работоспособности системы были проведены следующие виды тестирования:

\begin{enumerate}[label=\arabic*), leftmargin=*, itemsep=0pt]
\item \textbf{Функциональное тестирование} - проверка всех функций системы согласно требованиям
\item \textbf{Тестирование граничных условий} - работа с минимальными и максимальными значениями
\item \textbf{Тестирование обработки ошибок} - проверка реакции на некорректные данные
\item \textbf{Интеграционное тестирование} - проверка взаимодействия компонентов системы
\item \textbf{Тестирование восстановления данных} - проверка загрузки сохраненного состояния
\end{enumerate}

\subsection{Результаты тестирования}

В результате тестирования подтверждена работоспособность всех основных функций системы:

\begin{table}[H]
\centering
\caption{Результаты тестирования системы}
\label{table:test_results}
\begin{tblr}[
  theme = vvsu,
  ]{
    width = 0.95\linewidth,
    colspec = {|X[1,l]|X[2,l]|X[1,c]|},
  }
Функция & Описание теста & Результат \\
Обслуживание клиента & Продажа 20 литров АИ-95 & Успешно \\
Проверка цистерн & Отображение состояния всех цистерн & Успешно \\
Пополнение топлива & Заправка цистерны на 5000 литров & Успешно \\
Статистика & Формирование отчета по продажам & Успешно \\
История операций & Просмотр последних транзакций & Успешно \\
Перекачка топлива & Перемещение 1000 литров между цистернами & Успешно \\
Аварийный режим & Активация и деактивация аварийного режима & Успешно \\
Сохранение данных & Сохранение и восстановление состояния & Успешно \\
\end{tblr}
\end{table}

\subsection{Обработка исключительных ситуаций}

Система корректно обрабатывает следующие исключительные ситуации:

\begin{enumerate}[label=\arabic*), leftmargin=*, itemsep=0pt]
\item Попытка продажи при отсутствии достаточного количества топлива
\item Попытка пополнения сверх максимальной емкости цистерны
\item Выбор несуществующей колонки или типа топлива
\item Ввод некорректных числовых значений
\item Отсутствие файла данных при запуске системы
\item Попытка перекачки топлива между цистернами разных типов
\item Активация аварийного режима при работающих операциях
\end{enumerate}

\begin{conclusion}
В ходе выполнения проекта была разработана полнофункциональная консольная система управления автозаправочной станцией, соответствующая всем требованиям технического задания. Система успешно решает следующие задачи:

\begin{enumerate}[label=\arabic*), leftmargin=*, itemsep=0pt]
\item Автоматизация процесса обслуживания клиентов с расчетом стоимости и учетом топлива
\item Эффективное управление запасами топлива в цистернах с автоматическим контролем уровня
\item Ведение полной статистики по всем аспектам работы АЗС
\item Обеспечение безопасности через аварийный режим и резервное копирование данных
\item Сохранение состояния системы между сеансами работы
\end{enumerate}

Разработанная система демонстрирует следующие преимущества:
\begin{enumerate}[label=\arabic*), leftmargin=*, itemsep=0pt]
\item Удобный консольный интерфейс с пошаговым руководством пользователя
\item Надежное хранение данных в формате JSON
\item Модульная архитектура, позволяющая легко расширять функциональность
\item Полная обработка исключительных ситуаций
\item Соответствие реальным бизнес-процессам АЗС
\end{enumerate}

Система может быть использована в учебных целях для демонстрации принципов объектно-ориентированного программирования, работы с файлами и построения консольных приложений. При необходимости система может быть расширена добавлением графического интерфейса, сетевых функций или интеграции с базами данных.

Перспективы развития системы включают:
\begin{enumerate}[label=\arabic*), leftmargin=*, itemsep=0pt]
\item Добавление модуля анализа эффективности работы АЗС
\item Реализация функций планирования поставок топлива
\item Интеграция с системами учета и отчетности
\item Разработка веб-интерфейса для удаленного управления
\item Добавление многопользовательского режима с разграничением прав доступа
\end{enumerate}

Разработанная система является законченным программным продуктом, готовым к использованию в учебном процессе и в качестве основы для дальнейшего развития.
\end{conclusion}

\references

\begin{application}{Приложение А}{Исходный код программы}
\label{application:source_code}
Основной модуль программы (main.py) содержит полную реализацию системы управления АЗС. Код структурирован и документирован в соответствии с требованиями к промышленному программированию.

Программа состоит из следующих основных компонентов:

\begin{enumerate}[label=\arabic*), leftmargin=*, itemsep=0pt]
\item Константы и настройки системы (типы топлива, цены, схема подключения)
\item Класс \texttt{Cistern} для управления цистернами
\item Класс \texttt{FuelColumn} для управления заправочными колонками
\item Класс \texttt{Transaction} для учета операций
\item Класс \texttt{Statistics} для ведения статистики
\item Класс \texttt{GasStation} как основной класс системы
\item Функции для работы с меню и пользовательским интерфейсом
\item Главная функция \texttt{main()} с основным циклом программы
\end{enumerate}

Общий объем исходного кода составляет около 800 строк. Программа использует стандартные библиотеки Python: json, datetime, os, typing.
\end{application}

\begin{application}{Приложение Б}{Примеры работы системы}
\label{application:examples}
Пример сеанса работы с системой управления АЗС включает следующие этапы:

\begin{enumerate}[label=\arabic*), leftmargin=*, itemsep=0pt]
\item Запуск программы и загрузка сохраненного состояния
\item Отображение главного меню с доступными функциями
\item Выбор функции обслуживания клиента
\item Пошаговое выполнение операции продажи топлива:
\begin{itemize}[label=--, leftmargin=*]
\item Выбор заправочной колонки (от 1 до 8)
\item Выбор типа топлива из доступных на выбранной колонке
\item Ввод количества литров
\item Подтверждение операции
\item Отображение результата и возврат в главное меню
\end{itemize}
\item Использование других функций системы:
\begin{itemize}[label=--, leftmargin=*]
\item Просмотр состояния цистерн
\item Оформление пополнения запасов
\item Просмотр статистики и истории операций
\item Управление цистернами (включение/отключение)
\end{itemize}
\item Завершение работы с сохранением состояния
\end{enumerate}

Система обеспечивает интуитивно понятный интерфейс с подробными подсказками на каждом этапе работы. Все операции логируются и сохраняются для последующего анализа.
\end{application}

\end{document}